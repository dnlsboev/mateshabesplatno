%настройка документа
\documentclass[oneside, final]{book}
\usepackage[14pt]{extsizes}
\usepackage[utf8]{inputenc}
\usepackage[russian]{babel}
\frenchspacing

\binoppenalty=10000 
\relpenalty=10000 

%настройка страницы
\usepackage{vmargin}
\setpapersize{A4}
\setmarginsrb{2cm}{1.5cm}{1cm}{1.5cm}{1cm}{1cm}{0pt}{13mm}

%настройка переносов и красной строки
\usepackage{indentfirst}
\sloppy
%\fussy

%пакеты AMS
\usepackage{amsmath}
\usepackage{amsfonts}
\usepackage{amssymb}
\usepackage{amsthm}

%дополнительные команды
\newcommand{\mbb}[1]{\mathbb{#1}}
\newcommand{\mcl}[1]{\mathcal{#1}}
\newcommand{\mfrk}[1]{\mathfrak{#1}}

\newcommand{\mylim}[1]{\lim\limits_{#1 \to \infty}}
\newcommand{\mysum}[3]{\sum\limits_{#1 = #2}^#3}
\newcommand{\mysup}[1]{\underset{#1 \in \mbb N}{\sup}}
\newcommand{\myinf}[1]{\underset{#1 \in \mbb N}{\inf}}

%теоремы и тд с двойной нумерацией
\theoremstyle{plain}
\newtheorem{theorem}{Теорема}[chapter]
\newtheorem{lemma}[theorem]{Лемма}
\newtheorem{corollary}[theorem]{Следствие}
\newtheorem{statement}[theorem]{Утверждение}
\newtheorem{remark}[theorem]{Замечание}

\theoremstyle{definition}
\newtheorem{mdef}{Определение}[chapter]

%окружение доказательства
\newenvironment{Proof}[1][Доказательство] % имя окружения 
{\par\noindent{\bf #1. }} % команды для \begin 
{\hfill $\scriptstyle\qed$} % команды для \end 

\title{\textbf{Функции нескольких переменных}\\ Основные теоремы (без доказательств)}
\author{Математический анализ, 3 семестр}
\date{ММФ НГУ\\ \vfill2019 год}

\begin{document}
	\maketitle

	\tableofcontents
	
	\chapter{Нормированные векторные пространства}	
	\begin{mdef}
		$\mcl V$ -- в. п-во над $\mbb R$, если
		\begin{align}
			&\forall x, y \in \mcl V \ x+y = y+x, \tag{\text{ВП.1}}\\
			&\forall x,y, z \in \mcl V \ (x+y)+z=x+(y+z), \tag{\text{ВП.2}}\\
			&\exists \vec 0\ \forall x \in \mcl V\ \vec 0 + x = x + \vec 0 = x, \tag{\text{ВП.3}}\\
			&\forall x \in \mcl V\ \exists -x \in \mcl V: \ x+(-x) = \vec 0, \tag{\text{ВП.4}}\\
			&\forall \alpha, \beta \in \mbb R\ \forall x \in \mcl V \ (\alpha+\beta)x = \alpha x+\beta x, \tag{\text{ВП.5}}\\
			&\forall \alpha, \beta \in \mbb R\ \forall x \in \mcl V\ (\alpha \beta) x = \alpha ( \beta x), \tag{\text{ВП.6}}\\
			&\forall \alpha \in \mbb R \ \forall x, y\in \mcl V\ \alpha(x+y) = \alpha x + \alpha y, \tag{\text{ВП.7}}\\
			&\exists 1 \in \mbb R \ \forall x \in \mcl V \ 1\cdot x = x. \tag{\text{ВП.8}}
		\end{align}
	\end{mdef}
	\begin{mdef}
		$\|\cdot\| \colon \mcl V \to \mbb R_{\ge 0}$ -- норма, если
		\begin{align}
			&\forall x\in \mcl V\ \|x\| \ge 0, \tag{\text{Н.1}}\\
			&\forall x, y \in \mcl V\ \|x+y\| \le \|x\|+\|y\|, \tag{\text{Н.2}}\\
			&\forall \alpha \in \mbb R \ \forall x \in \mcl V\ \|\alpha x \| = |\alpha|\|x\|, \tag{\text{Н.3}}\\
			&\|x\|=0 \Leftrightarrow x = \vec 0. \tag{\text{Н.4}}
		\end{align}
	\end{mdef}
	
	\begin{remark}
		На нормированном в. п-ве можно определить метрику следующим образом:
		$$
			\rho(x, y) = \|x-y\|.
		$$	
		
		И это действительно будет метрикой.
	\end{remark}

	\begin{mdef}
		Банахово пространство -- полное (относительно метрики из предыдущего замечания) нормированное векторное пространство.
	\end{mdef}

	\begin{mdef}
		$\left( \cdot, \cdot \right)\colon \mcl V \times \mcl V \to \mbb R$ -- скалярное произведение, если 
		\begin{align}
			&\forall x \in \mcl V\ (x, x) \ge 0, \tag{\text{СП.1}}\\
			&\forall x,y \in \mcl V \ (x,y) = (y,x), \tag{\text{СП.2}}\\
			&\forall \alpha, \beta \in \mbb R \ \forall x,y, z \in \mcl V\ (\alpha x + \beta y, z) = (\alpha x, z) + (\beta y, z), \tag{\text{СП.3}}\\
			&(x,x) = 0 \Leftrightarrow x = \vec 0. \tag{\text{СП.4}}
		\end{align}
	\end{mdef}

	\begin{mdef}
		Если $\mcl V$ -- в. пр-во и $(\cdot, \cdot)$ -- скалярное произведение, то $\left(\mcl V, (\cdot, \cdot)\right)$ -- евклидово пространство.
	\end{mdef}

	\begin{statement}
		В евклидовом пространстве можно ввести норму следующим образом:
		$$
			\|x\| = \sqrt{(x, x)}.
		$$
	\end{statement}

	\begin{mdef}
		Две нормы $\|\cdot\|_1$ и $\|\cdot\|_2$ билипшицево эквивалентны, если 
		$$
			\exists C > 0:\  \cfrac 1 C \|x\|_1 \le \|x\|_2 \le C \|x\|_1, \quad \forall x \in \mcl V 
		$$
	\end{mdef}

	\begin{theorem}
		Все нормы в $\mbb R ^n$ билипшицево эквивалентны.
	\end{theorem}
	\begin{corollary}
		Последовательность точек
		\begin{equation*}
			v^m = \begin{pmatrix}
			v^m_1 \\
			\vdots \\
			v^m_n
			\end{pmatrix}
		\end{equation*}
		сходится к точке 
		\begin{equation*}
			u = \begin{pmatrix}
				u_1 \\
				\vdots \\
				u_n
			\end{pmatrix}
		\end{equation*} 
		при $m\to\infty$ тогда и только тогда, когда $v_i^m \underset{m \to \infty}{\longrightarrow} u_i$ для всех $i=1,\ldots, n$.
	\end{corollary}

	\begin{mdef}
		Введем понятие отображения $F \colon \mbb R^n \to \mbb R^m$.
		$$
			F(x) = 	\begin{pmatrix}
				f_1(x) \\
				\vdots \\
				f_m(x)
			\end{pmatrix},
		$$ где $x \in U$ и $U$ -- область в $\mbb R^n$.
	\end{mdef}
	\begin{corollary}
		$$\lim \limits_{x\to a} F(x) = F(x) \Leftrightarrow \lim\limits_{\eta \to a_i} f_i(\eta) = f(a_i) \quad \text{для всех $i=1,\ldots, m$.} $$
	\end{corollary}

	\chapter{Дифференциальное исчисление функций нескольких переменных}
	
	\begin{mdef}
		Функция $f\colon \mcl D \to \mbb R$, где $\mcl D$ -- область в $\mbb R^n$, дифференцируема в точке $a \in \mcl D$, если 
		$$
			f(a+h)=f(a)+\sum\limits_{k=1}^n A_k h_k + o(\|h\|),
		$$
		где $A_k$ -- константы и $h = (h_1, \ldots, h_n)^T$.
		
		Аналогичное определение:
		$$
			f(a+h)=f(a)+\mcl L \left< h \right> + o(\|h\|).
		$$
		
		Нетрудно понять, что $\mcl L \left< h\right> = d f(a) \left< h \right>$ и 
		$$
			A_k = \left. \cfrac {\partial f(x)} {\partial x_k} \right|_{x=a} \quad \text{для всех $k=1,\ldots,n$}.
		$$
	\end{mdef}

	\begin{mdef}
		Пусть $\vec v$ -- некоторый фиксированный вектор из $\mbb R^n$.
		
		Производная функции $f$ в точке $a\in \mcl D$ по направлению $\vec v$ определяется следующим образом:
		$$
		\partial_{\vec v} f(a) = \lim \limits_{t\to0}\cfrac {f(a+t\vec v)-f(a)} t.
		$$
		
		Если положить $\varphi(t) = f(a+t\vec v)$, то
		$$
			\partial_{\vec v} f(a) = \cfrac {d\varphi(0)} {dt}.
		$$
		
		Если $\vec v = \vec {e_i}$, то
		$$
			\partial_{\vec{e_i}} f(a) = \cfrac {\partial f(a)} {\partial x_i} = f'_{x_i}(a). 
		$$
	\end{mdef}

	\begin{statement}
		Если функция $f$ дифференцируема в точке $a\in \mcl D$, то для любого вектора $\vec v \in \mbb R^n$ существует $\partial_{\vec v}f(a)$,
		которую можно вычислить следующими способами:
		\begin{align*}
			&\partial_{\vec v} f(a) = \left((f'_{x_1}(a), \ldots, f'_{x_n}(a))^T, (v_1, \ldots, v_n)^T\right),\\
			&\partial_{\vec v}f(a) = \left(\text{grad}f(a), \vec v\right)
		\end{align*}
	\end{statement}

	\begin{lemma}
		Пусть функция $f\colon \mcl D \to \mbb R$, где $\mcl D$ -- выпуклая область в $\mbb R^n$, имеет в $\mcl D$ ограниченные константой $K$ частные производные, тогда 
		$$
			\forall a, b\in \mcl D \quad |f(b)-f(a)|\le n K \|b-a\|_2
		$$
		или 
		$$
			\forall a,b \in \mcl D \quad |f(b)-f(a)|\le K \|b-a\|_1.
		$$
	\end{lemma}

	\begin{remark}
		Далее полагается, что множество определения функций -- выпуклая область, если не оговорено другое.
	\end{remark}

	\begin{theorem}
		Пусть все частные производные функции $f\colon \mcl D \to \mbb R$ существуют в $\mcl D$ и непрерывны в точке $a\in\mcl D$, тогда $f$ дифференцируема в точке $a$.
	\end{theorem}

	\begin{theorem}
		Пусть $f, g \in D(a)$. 
		
		Тогда $f+g, fg \in D(a)$ 
		\begin{align*}
			&d(f+g)(a) = df(a)+dg(a),\\
			&d(fg)(a)=g(a)df(a)+f(a)dg(a).
		\end{align*}
	\end{theorem}
	
	\begin{mdef}
		Вектор-функция (векторное поле) $f\colon U \to \mbb R^m$, где $U \subset \mbb R^n$, дифференцируема в точке $a\in U$, если 	
		$$
			f(a+h)=f(a)+df(a)\left< h\right> +o(\|h\|).
		$$
	\end{mdef}

	\begin{mdef}
		Введем понятие матрицы Якоби отображения $f\colon U \to \mbb R^m$, где $U \subset \mbb R^n$, в точке $a\in U$:
		$$
			\cfrac {\partial f(a)} {\partial(x_1, \ldots, x_n)} = \left. \cfrac {\partial f(x)} {\partial(x_1, \ldots, x_n)} \right|_{x=a} = \begin{pmatrix}
				\cfrac {\partial f_1(a)} {\partial x_1} & \dots & \cfrac {\partial f_1(x)} {\partial x_n} \\
				\vdots & \ddots & \vdots \\
				\cfrac {\partial f_m(a)} {\partial x _1} & \dots & \cfrac {\partial f_m(a)} {\partial x_n}
			\end{pmatrix}.
		$$
		и понятие Якобиана отображения в точке $a \in U$:
		$$
			\mcl J = \left|  \cfrac {\partial f(a)} {\partial(x_1, \ldots, x_n)}  \right|.
		$$
	
		Тогда $df(a) = \cfrac {\partial f(a)} {\partial(x_1, \ldots, x_n)}$.
	\end{mdef}
	
	\begin{theorem}
		Если функция $f\colon U \to \mbb R^m$, где $U \subset \mbb R^n$, дифференцируема в точке $a\in U$, и функция $g \colon V \to \mbb R^k$, где $V \subset R^m$, дифференцируема в точке $f(a)$, тогда
		$g \circ f$ дифференцируема в точке $a \in U$ и 
		$$
			d(g \circ f)(a) = dg(f(a))\cdot df(a).
		$$
	\end{theorem}
	\begin{corollary}
		Пусть отображение $f\colon U \to \mbb R^n$, где $U \subset \mbb R^n$, дифференцируемо в точке $a\in U$ и  имеет обратное отображение $f^{-1}$.
		
		Тогда $f^{-1} \in D(f(a))$ и 
		$$
			df^{-1}(f(a)) = (df(a))^{-1}.
		$$
	\end{corollary}
	
	
	\section{Производные высших порядков}
	
	\section{Формула Тейлора}

	\section{Дифференциалы высших порядков}
	
	\section{Экстремумы функций нескольких переменных}
	
	\section{Теорема о неявной функции}
	
	\section{Диффеоморфизмы} 
		
	\chapter{Многообразия}
	
	\begin{thebibliography}{0}
		\bibitem{Spivak} М.~Спивак, Математический анализ на многобразиях
	\end{thebibliography}
\end{document}
