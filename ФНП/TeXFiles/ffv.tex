\chapter{Дифференциальное исчисление функций нескольких переменных}
	
	\begin{mdef}
		Функция $f\colon \mcl D \to \mbb R$, где $\mcl D$ -- область в $\mbb R^n$, дифференцируема в точке $a \in \mcl D$, если 
		$$
			f(a+h)=f(a)+\sum\limits_{k=1}^n A_k h_k + o(\|h\|),
		$$
		где $A_k$ -- константы и $h = (h_1, \ldots, h_n)^T$.
		
		Аналогичное определение:
		$$
			f(a+h)=f(a)+\mcl L \left< h \right> + o(\|h\|),
		$$
		где $\mcl L$ -- линейное отображение.
		
		Нетрудно понять, что $\mcl L \left< h\right> = d f(a) \left< h \right>$ и 
		$$
			A_k = \left. \cfrac {\partial f(x)} {\partial x_k} \right|_{x=a} \quad \text{для всех $k=1,\ldots,n$}.
		$$
	\end{mdef}

	\begin{mdef}
		Пусть $\vec v$ -- некоторый фиксированный вектор из $\mbb R^n$.
		
		Производная функции $f$ в точке $a\in \mcl D$ по направлению $\vec v$ определяется следующим образом:
		$$
		\partial_{\vec v} f(a) = \lim \limits_{t\to0}\cfrac {f(a+t\vec v)-f(a)} t.
		$$
		
		Если положить $\varphi(t) = f(a+t\vec v)$, то
		$$
			\partial_{\vec v} f(a) = \cfrac {d\varphi(0)} {dt}.
		$$
		
		Если $\vec v = \vec {e_i}$, то
		$$
			\partial_{\vec{e_i}} f(a) = \cfrac {\partial f(a)} {\partial x_i} = f'_{x_i}(a). 
		$$
	\end{mdef}

	\begin{statement}
		Если функция $f$ дифференцируема в точке $a\in \mcl D$, то для любого вектора $\vec v \in \mbb R^n$ существует $\partial_{\vec v}f(a)$,
		которую можно вычислить следующими способами:
		\begin{align*}
			&\partial_{\vec v} f(a) = \left((f'_{x_1}(a), \ldots, f'_{x_n}(a))^T, (v_1, \ldots, v_n)^T\right),\\
			&\partial_{\vec v}f(a) = \left(\text{grad}f(a), \vec v\right)
		\end{align*}
	\end{statement}

	\begin{lemma}
		Пусть функция $f\colon \mcl D \to \mbb R$, где $\mcl D$ -- выпуклая область в $\mbb R^n$, имеет в $\mcl D$ ограниченные константой $K$ частные производные, тогда 
		$$
			\forall a, b\in \mcl D \quad |f(b)-f(a)|\le n K \|b-a\|_2
		$$
		или 
		$$
			\forall a,b \in \mcl D \quad |f(b)-f(a)|\le K \|b-a\|_1.
		$$
	\end{lemma}

	\begin{remark}
		Далее полагается, что множество определения функций -- выпуклая область, если не оговорено другое.
	\end{remark}

	\begin{theorem}
		Пусть все частные производные функции $f\colon \mcl D \to \mbb R$ существуют в $\mcl D$ и непрерывны в точке $a\in\mcl D$, тогда $f$ дифференцируема в точке $a$.
	\end{theorem}

	\begin{theorem}
		Пусть $f, g \in D(a)$. 
		
		Тогда $f+g, fg \in D(a)$ 
		\begin{align*}
			&d(f+g)(a) = df(a)+dg(a),\\
			&d(fg)(a)=g(a)df(a)+f(a)dg(a).
		\end{align*}
	\end{theorem}
	
	\begin{mdef}
		Вектор-функция (векторное поле) $f\colon U \to \mbb R^m$, где $U \subset \mbb R^n$, дифференцируема в точке $a\in U$, если 	
		$$
			f(a+h)=f(a)+df(a)\left< h\right> +o(\|h\|).
		$$
	\end{mdef}

	\begin{mdef}
		Введем понятие матрицы Якоби отображения $f\colon U \to \mbb R^m$, где $U \subset \mbb R^n$, в точке $a\in U$:
		$$
			\cfrac {\partial f(a)} {\partial(x_1, \ldots, x_n)} = \left. \cfrac {\partial f(x)} {\partial(x_1, \ldots, x_n)} \right|_{x=a} = \begin{pmatrix}
				\cfrac {\partial f_1(a)} {\partial x_1} & \dots & \cfrac {\partial f_1(x)} {\partial x_n} \\
				\vdots & \ddots & \vdots \\
				\cfrac {\partial f_m(a)} {\partial x _1} & \dots & \cfrac {\partial f_m(a)} {\partial x_n}
			\end{pmatrix}.
		$$
		и понятие Якобиана отображения в точке $a \in U$:
		$$
			\mcl J_f(a) = \left|  \cfrac {\partial f(a)} {\partial(x_1, \ldots, x_n)}  \right|.
		$$
	\end{mdef}
	
	\begin{theorem}
		Если функция $f\colon U \to \mbb R^m$, где $U \subset \mbb R^n$, дифференцируема в точке $a\in U$, и функция $g \colon V \to \mbb R^k$, где $V \subset R^m$, дифференцируема в точке $f(a)$, тогда
		$g \circ f$ дифференцируема в точке $a \in U$ и 
		$$
			d(g \circ f)(a) = dg(f(a))\cdot df(a).
		$$
	\end{theorem}
	\begin{corollary}
		Пусть отображение $f\colon U \to \mbb R^n$, где $U \subset \mbb R^n$, дифференцируемо в точке $a\in U$ и  имеет обратное отображение $f^{-1}$.
		
		Тогда $f^{-1} \in D(f(a))$ и 
		$$
			df^{-1}(f(a)) = (df(a))^{-1}.
		$$
	\end{corollary}
	
	
	\section{Производные высших порядков}
	\begin{mdef}
		Функция $f$ $k$-раз дифференцируема в точке $a \in \mcl D$, если любая $n$-ая частная производная, где $n=1,\ldots,k-1$, дифференцируема в точке $a \in \mcl D$.
	\end{mdef}
	\begin{remark}
		Используют также понятие гладкости функции $f$. 
		
		Функция $f$ класса $\mcl C^r$ в точке $a\in \mcl D$, если каждая ее $k$-ая производная, где $k=1,\ldots,r$, непрерывна.
		
		По достаточному условию дифференцируемости, функция $f \in D^{r}$.
	\end{remark}

	\begin{theorem}\label{th1}
		Пусть $u = f(x,y) \in D^2(M_0)$, $M_0 = (x_0, y_0) \in U$ и $M_0$ -- внутренняя точка $U$.
		Тогда
		$$
		\frac {\partial^2 u(M_0)} {\partial x \partial y} = \frac {\partial^2 u(M_0)} {\partial y \partial x}.
		$$
	\end{theorem}

	\begin{theorem}
		Пусть $f\colon U \to \mbb R^2$ в некоторой окрестности точки $M_0 = (x_0, y_0)$ имеет частные производные $f'_x, f'_y$, $f''_{xy}, f''_{yx}$ и при этом вторые производные непрерывны в $M_0$.
		Тогда
		$$
		f''_{xy}(M_0) = f''_{yx}(M_0).
		$$
	\end{theorem}

	\begin{corollary}
		Пусть $u = f(x_1, \ldots, x_n)$ -- $m$-раз дифференцируемая функция в точке $M_0$, при этом $M_0 = (x_1^0, \ldots, x_n^0)$.
		Тогда в точке $M_0$ любые смешанные производные равны, то есть
		$$
		\frac {\partial^k f(M_0)} {\partial x_{\pi(1)} \ldots \partial x_{\pi(k)}} = \frac {\partial^k f(M_0)} {\partial x_{\sigma(1)} \ldots \partial x_{\sigma(k)}},
		$$
		где $\pi, \sigma \in \mbb S_k$ такие, что $\pi \not= \sigma$.
	\end{corollary}
	\section{Формула Тейлора}
	\begin{mdef}
		Определим понятие мультииндекса.
		
		$$
			\alpha = (\alpha_1, \ldots, \alpha_n),
		$$
		где $\alpha_i \in \mbb N_{\ge 0}$ для всех $i \in \mbb N$.
		
		Так же определим следующие операции:
		\begin{align*}
			&\alpha! = \alpha_1!\dots \alpha_n!,\\
			&|\alpha| = \sum\limits_{k=1}^n \alpha_k,\\
			&\text{Пусть $h \in \mcl V: h = (h_1, \ldots, h_n)^T$, то } h^{\alpha} = \prod \limits_{i=1}^n h_i^{\alpha_i}. 
		\end{align*}
		
		Удобство мультииндекса заключается в компактности обозначений. Далее под 
		$$
			\cfrac {\partial^{\alpha}f} {\partial x^{\alpha}}
		$$	
		будем подразумевать
		$$
			\cfrac {\partial^{\alpha_1}} {\partial x_1^{\alpha_1}} \left(\cfrac {\partial^{\alpha_2}} {\partial x_2^{\alpha_2}}\dots \left(\cfrac {\partial^{\alpha_n}f} {\partial x_n^{\alpha_n}}\right)\dots\right).
		$$
	\end{mdef}

	\begin{theorem}
		$$
			(x_1 + \ldots + x_n)^k = \sum \limits_{\substack{\alpha_i \ge 0,\ i=1,\ldots,n \\ \alpha_1+\dots\alpha_n=k}} \cfrac {|\alpha|!}{\alpha!} \prod\limits_{i=1}^n x_i^{\alpha_i}.
		$$
	\end{theorem}

	\begin{example}
		Обозначим за $\mfrk D$ следующее отображение:
		$$
			\mfrk D = \sum\limits_{i=1}^n h_i \cfrac {\partial} {\partial x_i},
		$$
		где $h_i$ -- $i$-ая компонента фиксированного вектора $h = (h_1, \ldots, h_n)^T$.
		
		Если функция $f \in D^k$, то 
		\begin{multline*}
			\mfrk{D}^k f = \left( \sum\limits_{i=1}^n h_i \cfrac{\partial} {\partial x_i} \right)^k f =\\
			= \left( \sum\limits_{\substack{\alpha_i \ge 0,\ i=1,\ldots,n \\ \alpha_1+\dots\alpha_n=k}} \cfrac {k!}{\alpha_1!\dots \alpha_n!} \prod\limits_{i=1}^n h_i^{\alpha_i}\cfrac {\partial^{\alpha_i}}{\partial x_i^{\alpha_i}} \right)f =\\
			= \left(\sum\limits_{|\alpha|=k} \cfrac{|\alpha|!}{\alpha!} h^{\alpha} \cfrac {\partial^{\alpha}}{\partial x^{\alpha}}\right) f.
		\end{multline*}
	\end{example}

	\begin{theorem}[Формула Тейлора с остаточным членом в форме Пеано]
		Пусть $f\in D^k(a)$, где $a\in U \subset \mbb R^n$.
		
		Тогда
		$$
			f(a+h) = \sum\limits_{i=0}^k \left( \sum\limits_{|\alpha|=i} \cfrac {1}{\alpha!}  \cfrac {\partial^\alpha f(a)}{\partial x^\alpha} h^\alpha\right) + o(\|h\|^k),
		$$
		
		где $o(\|h\|^k) \to 0$ при $\|h\|\to 0$.
	\end{theorem}

	\begin{mdef}
		$
		\sum\limits_{i=0}^k \left( \sum\limits_{|\alpha|=i} \cfrac {1}{\alpha!}  \cfrac {\partial^\alpha f(a)}{\partial x^\alpha} h^\alpha\right)
		$ -- полином Тейлора порядка $k$ в точке $a\in U$ функции $f\in D^k(a)$.
	\end{mdef}

	\begin{lemma}
		$h^\alpha = o(\|h\|^k)$ при $\|h\|\to0$ тогда и только тогда, когда $k < |\alpha|$.
	\end{lemma}

	\begin{theorem}[Единственность полинома Тейлора]
		Пусть существует полином 
		$$
			P = \sum \limits_{i=0}^k \sum\limits_{|\alpha|=i} P_\alpha h^\alpha
		$$
		такой, что 
		$$
			f(a+h) - P = o(\|h\|^k)
		$$ при $\|h\|\to 0$.
		
		Тогда $P$ -- полином Тейлора.
	\end{theorem}
	
	\begin{corollary}
		Пусть $f\in D^k(a)$ и $$\cfrac {\partial^\alpha f(a)} {\partial x^\alpha} = 0$$ при $|\alpha| \le k$.
		
		Тогда $f(a+h) = o(\|h\|^k)$ при $\|h\|\to0$.
	\end{corollary}
	
	\section{Дифференциалы высших порядков}
	Пусть $f = f(x_1, \ldots, x_n)$ -- дифференцируемая функция.
	Тогда дифференциал функции $f$ имеет вид
	$$
		df = \cfrac {\partial f} {\partial x_1} dx_1 + \dots + \cfrac {\partial f} {\partial x_n}.	
	$$
	
	Обозначим за $\delta$ приращение.
	\begin{mdef}
		Значение $\delta(df)$ при $\delta x_i = dx_i$ для всех $i = 1,\ldots,n$ называется вторым дифференциалом функции $f$ и обозначается $d^2 f$.
		$$
			d^2 f = \delta \left. \left( \sum \limits_{i=1}^n \cfrac {\partial f}{\partial x_i}dx_i \right) \right|_{\substack{\delta x_i = dx_i, \\ i = 1,\ldots, n}}
		$$ 	
	\end{mdef}
	\begin{mdef}
		$k$-кратный дифференциал функции $f$ определяется следующим образом:
		$$
			\left.\delta(d^{k-1}f)\right|_{\substack{\delta x_i = dx_i, \\ i = 1,\ldots, n}} = d^k f.
		$$
	\end{mdef}

	\begin{theorem}
		Пусть функция $f=f(x_1, \ldots, x_n)$ дважды дифференцируема, тогда ее второй дифференциал имеет вид
		\begin{enumerate}
			\item Если $x_1, \ldots, x_n$ -- независимые переменные, то
			$$
				d^2f = \sum\limits_{i=1}^n \sum\limits_{j=1}^n \cfrac {\partial^2 f} {\partial x_j \partial x_i} dx_j dx_i.
			$$
			\item Если $x_i = x_i(\xi_1, \ldots, \xi_m)$ для всех $i = 1,\ldots, n$ дважды дифференцируемые функции, и $f(x_1, \ldots, x_n)$ -- дважды дифференцируемая функция, тогда
			$$
				d^2 f = \sum\limits_{i=1}^n \sum\limits_{j=1}^n \cfrac {\partial^2 f} {\partial x_j \partial x_i} dx_j dx_i +  \sum\limits_{k=1}^n \cfrac {\partial f} {\partial x_k} \sum\limits_{i=1}^m \sum\limits_{j=1}^m \cfrac {\partial^2 x_k} {\partial \xi_j \partial \xi_i} d\xi_j d\xi_i,
			$$ в более компактной форме:
			$$
				d^2 f = \left(\cfrac {\partial f} {\partial x_1} dx_1 + \dots +\cfrac {\partial f} {\partial x_n} dx_n\right)^2 + \sum \limits_{k=1}^n \cfrac {\partial f} {\partial x_k} d^2 x_k.
			$$
		\end{enumerate}
	\end{theorem}
	\begin{remark}
		Второй дифференциал функции $f$ от независимых переменных $x_1, \ldots, x_n$ $$d^2 f = \left(\cfrac {\partial f} {\partial x_1} dx_1 + \dots +\cfrac {\partial f} {\partial x_n} dx_n\right)^2$$ является симметричной квадратичной формой.
	\end{remark}

	\begin{statement} Пусть функция $f\colon U \to \mbb R$ $k$ раз дифференцируема в точке $a\in U$. Тогда 
		$$
		d^k f(a) = \sum_{\left| \alpha \right| = k} \cfrac{k!}{\alpha!} \cfrac{ \partial^{\left| \alpha \right|}f}{\partial x^{\alpha}}(a)(dx)^{\alpha}, \quad \text{где $(dx)^{\alpha} = (d{x_1})^{\alpha_1} \cdots (d{x_n})^{\alpha_n}$}
		$$
	\end{statement}
	
	\begin{corollary} Пусть функция $f\colon U \to \mbb R$ $k$ раз дифференцируема в точке $a\in U$. Тогда
		$$
		f(a+h) = f(a) + \cfrac 1 {1!}d f(a)\left< h\right> + \cfrac1 {2!} d^2 f(a) {\left< h \right>}^2 + \dots + \cfrac1{k!} d^k f(a) {\left< h \right>}^k + o(\| h \|^k)
		$$
	\end{corollary}
	
	\section{Экстремумы функций нескольких переменных}
	
	\section{Теорема о неявной функции}
	Для функции $F\colon U \to \mbb{R}^m$, где $U\subset \mbb{R}^{n+m} = \mbb{R}^n \times \mbb{R}^m$, то есть запись $(a, b) \in U$ означает, что $a\in \mbb R^n$, $b \in \mbb R^m$.
	Определим $F_a\colon U_a \to \mbb R^m$, где $U_a = \{y\in\mbb R^m\mid (a,y)\in U\}$, то есть $F_a(y) = F(a,y)$ и $F_b\colon U_b \to \mbb R^m$ аналогично ($F_b(x) = F(x, b)$).
	
	\begin{theorem}[О неявной функции]
		Пусть $F\colon U \to \mbb{R}^m$, где $U\subset \mbb{R}^{n+m}$, $U\in \mcl N[(a,b)]$ -- некоторая окрестность точки $(a,b)$. 
		
		Если выполнены следующие условия:
		\begin{enumerate}
			\item $F(a,b)=0$,
			\item \label{cond1} $F\in \mcl C^1 [(a,b)]$, то есть $F$ непрерывно дифференцируема в некоторой окрестности точки $(a,b)$,
			\item $dF_a(b)$ -- невырожденный оператор ($\det(dF_a(b))\not= 0$),
		\end{enumerate}
		то найдутся окрестности $V \in \mcl N(a)$, $V \subset \mbb R^n$, и $W \in \mcl N(b)$, $W \subset \mbb R^m$, и функция $f\colon V \to\mbb R^m$, $f\in\mcl D(V)$, что
		\begin{enumerate}
			\item $F(x,y) = 0 \Longleftrightarrow y=f(x)$ для $x\in V, y \in W$,
			\item $df(x) = - (dF_x(y))^{-1}dF_y(x)$, где $y = f(x)$.
		\end{enumerate}
	\end{theorem}	

	\begin{corollary}[О гладкости неявной функции]
		Если при выполнении условий теоремы о неявной функции потребовать, что бы $F\in\mcl C^r[(a,b)]$, то и неявная функция $f$ будет принадлежать классу $\mcl C^r$ в области определения $V$.
	\end{corollary}

	\begin{corollary}[Теорема об обратной функции]
		Пусть $f\colon U \to \mbb R^n$, $U\subset \mbb R^n$, $U \in \mcl N (b)$, $b\in \mbb R^n$. 
		
		Если верно, что 
		\begin{enumerate}
			\item $f\in \mcl C^1(b)$,
			\item $df(b)$ -- невырожденный оператор,
		\end{enumerate}
		то найдется $V\in \mcl N(f(b))$, $V \subset \mbb R^n$ в которой определена обратная функция $f^{-1}\colon V \to U$ такая, что $f^{-1}\in \mcl D(V)$ и $df^{-1}(x) = (df(y))^{-1}$, где $x=f(y)$.
	\end{corollary}
	
	\section{Диффеоморфизмы}
	
	\section{Условные экстремумы} 