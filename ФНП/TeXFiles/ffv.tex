\chapter{Дифференциальное исчисление функций нескольких переменных}
	
	\begin{mdef}
		Функция $f\colon \mcl D \to \mbb R$, где $\mcl D$ -- область в $\mbb R^n$, дифференцируема в точке $a \in \mcl D$, если 
		$$
			f(a+h)=f(a)+\sum\limits_{k=1}^n A_k h_k + o(\|h\|),
		$$
		где $A_k$ -- константы и $h = (h_1, \ldots, h_n)^T$.
		
		Аналогичное определение:
		$$
			f(a+h)=f(a)+\mcl L \left< h \right> + o(\|h\|),
		$$
		где $\mcl L$ -- линейное отображение.
		
		Нетрудно понять, что $\mcl L \left< h\right> = d f(a) \left< h \right>$ и 
		$$
			A_k = \left. \cfrac {\partial f(x)} {\partial x_k} \right|_{x=a} \quad \text{для всех $k=1,\ldots,n$}.
		$$
	\end{mdef}

	\begin{mdef}
		Пусть $\vec v$ -- некоторый фиксированный вектор из $\mbb R^n$.
		
		Производная функции $f$ в точке $a\in \mcl D$ по направлению $\vec v$ определяется следующим образом:
		$$
		\partial_{\vec v} f(a) = \lim \limits_{t\to0}\cfrac {f(a+t\vec v)-f(a)} t.
		$$
		
		Если положить $\varphi(t) = f(a+t\vec v)$, то
		$$
			\partial_{\vec v} f(a) = \cfrac {d\varphi(0)} {dt}.
		$$
		
		Если $\vec v = \vec {e_i}$, то
		$$
			\partial_{\vec{e_i}} f(a) = \cfrac {\partial f(a)} {\partial x_i} = f'_{x_i}(a). 
		$$
	\end{mdef}

	\begin{statement}
		Если функция $f$ дифференцируема в точке $a\in \mcl D$, то для любого вектора $\vec v \in \mbb R^n$ существует $\partial_{\vec v}f(a)$,
		которую можно вычислить следующими способами:
		\begin{align*}
			&\partial_{\vec v} f(a) = \left((f'_{x_1}(a), \ldots, f'_{x_n}(a))^T, (v_1, \ldots, v_n)^T\right),\\
			&\partial_{\vec v}f(a) = \left(\text{grad}f(a), \vec v\right)
		\end{align*}
	\end{statement}

	\begin{lemma}
		Пусть функция $f\colon \mcl D \to \mbb R$, где $\mcl D$ -- выпуклая область в $\mbb R^n$, имеет в $\mcl D$ ограниченные константой $K$ частные производные, тогда 
		$$
			\forall a, b\in \mcl D \quad |f(b)-f(a)|\le n K \|b-a\|_2
		$$
		или 
		$$
			\forall a,b \in \mcl D \quad |f(b)-f(a)|\le K \|b-a\|_1.
		$$
	\end{lemma}

	\begin{remark}
		Далее полагается, что множество определения функций -- выпуклая область, если не оговорено другое.
	\end{remark}

	\begin{theorem}
		Пусть все частные производные функции $f\colon \mcl D \to \mbb R$ существуют в $\mcl D$ и непрерывны в точке $a\in\mcl D$, тогда $f$ дифференцируема в точке $a$.
	\end{theorem}

	\begin{theorem}
		Пусть $f, g \in D(a)$. 
		
		Тогда $f+g, fg \in D(a)$ 
		\begin{align*}
			&d(f+g)(a) = df(a)+dg(a),\\
			&d(fg)(a)=g(a)df(a)+f(a)dg(a).
		\end{align*}
	\end{theorem}
	
	\begin{mdef}
		Вектор-функция (векторное поле) $f\colon U \to \mbb R^m$, где $U \subset \mbb R^n$, дифференцируема в точке $a\in U$, если 	
		$$
			f(a+h)=f(a)+df(a)\left< h\right> +o(\|h\|).
		$$
	\end{mdef}

	\begin{mdef}
		Введем понятие матрицы Якоби отображения $f\colon U \to \mbb R^m$, где $U \subset \mbb R^n$, в точке $a\in U$:
		$$
			\cfrac {\partial f(a)} {\partial(x_1, \ldots, x_n)} = \left. \cfrac {\partial f(x)} {\partial(x_1, \ldots, x_n)} \right|_{x=a} = \begin{pmatrix}
				\cfrac {\partial f_1(a)} {\partial x_1} & \dots & \cfrac {\partial f_1(x)} {\partial x_n} \\
				\vdots & \ddots & \vdots \\
				\cfrac {\partial f_m(a)} {\partial x _1} & \dots & \cfrac {\partial f_m(a)} {\partial x_n}
			\end{pmatrix}.
		$$
		и понятие Якобиана отображения в точке $a \in U$:
		$$
			\mcl J_f(a) = \left|  \cfrac {\partial f(a)} {\partial(x_1, \ldots, x_n)}  \right|.
		$$
	\end{mdef}
	
	\begin{theorem}
		Если функция $f\colon U \to \mbb R^m$, где $U \subset \mbb R^n$, дифференцируема в точке $a\in U$, и функция $g \colon V \to \mbb R^k$, где $V \subset R^m$, дифференцируема в точке $f(a)$, тогда
		$g \circ f$ дифференцируема в точке $a \in U$ и 
		$$
			d(g \circ f)(a) = dg(f(a))\cdot df(a).
		$$
	\end{theorem}
	\begin{corollary}
		Пусть отображение $f\colon U \to \mbb R^n$, где $U \subset \mbb R^n$, дифференцируемо в точке $a\in U$ и  имеет обратное отображение $f^{-1}$.
		
		Тогда $f^{-1} \in D(f(a))$ и 
		$$
			df^{-1}(f(a)) = (df(a))^{-1}.
		$$
	\end{corollary}
	
	
	\section{Производные высших порядков}
	\begin{mdef}
		Функция $f$ $k$-раз дифференцируема в точке $a \in \mcl D$, если любая $n$-ая частная производная, где $n=1,\ldots,k-1$, дифференцируема в точке $a \in \mcl D$.
	\end{mdef}
	\begin{remark}
		Используют также понятие гладкости функции $f$. 
		
		Функция $f$ класса $\mcl C^r$ в точке $a\in \mcl D$, если каждая ее $k$-ая производная, где $k=1,\ldots,r$, непрерывна.
		
		По достаточному условию дифференцируемости, функция $f \in D^{r}$.
	\end{remark}

	\begin{theorem}\label{th1}
		Пусть $u = f(x,y) \in D^2(M_0)$, $M_0 = (x_0, y_0) \in U$ и $M_0$ -- внутренняя точка $U$.
		Тогда
		$$
		\frac {\partial^2 u(M_0)} {\partial x \partial y} = \frac {\partial^2 u(M_0)} {\partial y \partial x}.
		$$
	\end{theorem}
	\begin{Proof}
		По условию функция $u$ дважды дифференцируема, следовательно, $u'_x, u'_y$ определены в некоторой окрестности точки $M_0$ и сами являются дифференцируемыми.
		
		Рассмотрим функцию одной переменной 
		$$
		\Phi(h) = f(x_0+h, y_0+h)-f(x_0+h, y_0)-f(x_0, y_0+h)+f(x_0, y_0)
		$$ 
		и $\varphi(x) = f(x, y_0+h)-f(x, y_0)$.
		Тогда нетрудно проверить, что 
		\begin{equation}\label{delta phi}
		\Delta\varphi(x_0) = \varphi(x_0+h) - \varphi(x_0) = \Phi(h).
		\end{equation}
		
		Сама функция $\varphi(x)$ является дифференцируемой в некоторой окрестности $x_0$, так как исходная функция $f(x, y)$ является дифференцируемой (здесь можно совсем формализовать эти рассуждения и воспользоваться существованием производной $u'_x$ и определением дифференцирумой функции одной переменной).
		
		Мы можем применить теорему Лагранжа и получить
		\begin{equation}\label{Lagranz}
		\Delta \varphi(x_0) = \varphi'(x_0+\theta h)h = (f'_x(x_0+\theta h, y_0+h) - f'_x(x_0+\theta h, y_0))h, 
		\end{equation}
		где $\theta \in [0, 1]$.
		
		Преобразуем выражение (\ref{Lagranz}) далее:
		\begin{multline*}
		\left(f'_x(x_0+\theta h, y_0+h) - f'_x(x_0+\theta h, y_0)\right)h =\\
		= (f'_x(x_0+\theta h, y_0+h) - f'_x(x_0+\theta h, y_0) + f'_x(x_0, y_0) - f'_x(x_0, y_0))h =\\
		= (f'_x(x_0+\theta h, y_0+h) - f'_x(x_0, y_0))h -(f'_x(x_0+\theta h, y_0) - f'_x(x_0, y_0))h.
		\end{multline*}
		
		Рассмотрим $f'_x(x_0+\theta h, y_0+h) - f'_x(x_0, y_0)$. Так как функция $f'_x$ дифференцируема в точке $M_0$, то
		$$
		f'_x(x_0+\theta h, y_0+h) - f'_x(x_0, y_0) = f''_{xx}(M_0)\theta h + f''_{yx}(M_0)h + \alpha_1 \cdot \theta h + \beta_1  \cdot h,
		$$
		где $\alpha_1 = o(1)$ и $\beta_1 = o(1)$ при $h\to0$.
		
		Аналогично рассматриваем $f'_x(x_0+\theta h, y_0) - f'_x(x_0, y_0)$. 
		$$
		f'_x(x_0+\theta h, y_0) - f'_x(x_0, y_0) = f''_{xx}(M_0)\theta h + f''_{yx}(M_0)\cdot 0 + \alpha_2 \cdot \theta h,
		$$
		где $\alpha_2 = o(1)$ при $h\to 0$.
		
		Вернемся к равенству (\ref{delta phi}).
		\begin{multline*}
		\Phi = \Delta \varphi = (f''_{xx}(M_0)\theta h + f''_{yx}(M_0)h + \alpha_1 \cdot \theta h + \beta_1  \cdot h)h - \\
		- (f''_{xx}(M_0)\theta h + \alpha_2 \cdot \theta h)h = f''_{yx}(M_0)h^2 + \gamma \cdot h^2,
		\end{multline*}
		где $\gamma = o(1)$ при $h\to0$.
		
		Теперь проделаем точно такие же действия для функции одной переменной
		$$
		\psi(y) = f(x_0+h, y)-f(x_0, y).
		$$
		То есть, если в $\varphi(x)$ мы фиксировали вторую переменную ($y_0+h$ и $y_0$), то сейчас мы фиксируем первую.
		
		Получим следующее:
		$$
		\Phi = f''_{xy}(M_0)+\xi \cdot h^2 = f''_{yx}(M_0)h^2 + \gamma \cdot h^2 \underset{h\to0}{\longrightarrow} f''_{xy}(M_0) = f''_{xy}(M_0).
		$$
	\end{Proof}

	\begin{theorem}
		Пусть $f\colon U \to \mbb R^2$ в некоторой окрестности точки $M_0 = (x_0, y_0)$ имеет частные производные $f'_x, f'_y$, $f''_{xy}, f''_{yx}$ и при этом вторые производные непрерывны в $M_0$.
		Тогда
		$$
		f''_{xy}(M_0) = f''_{yx}(M_0).
		$$
	\end{theorem}
	\begin{Proof}
		Пусть $\Phi, \varphi, \psi$ те же самые функции, что в доказательстве теоремы \ref{th1}.
		
		Тогда 
		$$
		\Phi = \Delta\varphi = \varphi'(x_0+\theta h)h = (f'_x(x_0+\theta h, y_0+h) - f'_x(x_0+\theta h, y_0))h.
		$$
		
		В последнем выражении можно воспользоваться теоремой Лагранжа (по переменной $y$) и получить:
		$$
		\Phi = f''_{yx}(x_0 + \theta h, y_0 + \xi h)h^2,
		$$
		где $\theta, \xi \in [0,1]$.
		
		Аналогично, рассматривая функцию $\psi$ получаем
		$$
		\Phi = \Delta\psi = f''_{xy}(x_0+\eta h, y_0 + \zeta h)h^2,
		$$
		где $\eta, \zeta \in [0,1]$.
		
		При $h\to 0$ имеем 
		$$
		f''_{xy}(M_0) = f''_{yx}(M_0).
		$$
	\end{Proof}

	\begin{corollary}
		Пусть $u = f(x_1, \ldots, x_n)$ -- $m$-раз дифференцируемая функция в точке $M_0$, при этом $M_0 = (x_1^0, \ldots, x_n^0)$.
		Тогда в точке $M_0$ любые смешанные производные равны, то есть
		$$
		\frac {\partial^k f(M_0)} {\partial x_{\pi(1)} \ldots \partial x_{\pi(k)}} = \frac {\partial^k f(M_0)} {\partial x_{\sigma(1)} \ldots \partial x_{\sigma(k)}},
		$$
		где $\pi, \sigma \in \mbb S_k$ такие, что $\pi \not= \sigma$.
	\end{corollary}
	\begin{Proof}
		Достаточно заметить, что любую смешанную производную, меняя порядок дифференцирования по двум <<соседним>> переменным, можно перевести в смешанную производную с другим порядком дифференцирования. А возможность совершения таких действий обеспечивается теоремой \ref{th1}:
		$$
		\frac {\partial^k f(M_0)} {\partial x_{\pi(1)} \ldots (\partial x_{\pi(i)} \partial x_{\pi(i+1)}) \ldots \partial x_{\pi(k)}} = \frac {\partial^k f(M_0)} {\partial x_{\pi(1)} \ldots (\partial x_{\pi(i+1)} \partial x_{\pi(i)}) \ldots \partial x_{\pi(k)}}
		$$
	\end{Proof}
	\section{Формула Тейлора}
	\begin{mdef}
		Определим понятие мультииндекса.
		
		$$
			\alpha = (\alpha_1, \ldots, \alpha_n),
		$$
		где $\alpha_i \in \mbb N_{\ge 0}$ для всех $i \in \mbb N$.
		
		Так же определим следующие операции:
		\begin{align*}
			&\alpha! = \alpha_1!\dots \alpha_n!,\\
			&|\alpha| = \sum\limits_{k=1}^n \alpha_k,\\
			&\text{Пусть $h \in \mcl V: h = (h_1, \ldots, h_n)^T$, то } h^{\alpha} = \prod \limits_{i=1}^n h_i^{\alpha_i}. 
		\end{align*}
		
		Удобство мультииндекса заключается в компактности обозначений. Далее под 
		$$
			\cfrac {\partial^{\alpha}f} {\partial x^{\alpha}}
		$$	
		будем подразумевать
		$$
			\cfrac {\partial^{\alpha_1}} {\partial x_1^{\alpha_1}} \left(\cfrac {\partial^{\alpha_2}} {\partial x_2^{\alpha_2}}\dots \left(\cfrac {\partial^{\alpha_n}f} {\partial x_n^{\alpha_n}}\right)\dots\right).
		$$
	\end{mdef}

	\begin{theorem}
		$$
			(x_1 + \ldots + x_n)^k = \sum \limits_{\substack{\alpha_i \ge 0,\ i=1,\ldots,n \\ \alpha_1+\dots\alpha_n=k}} \cfrac {|\alpha|!}{\alpha!} \prod\limits_{i=1}^n x_i^{\alpha_i}.
		$$
	\end{theorem}

	\begin{example}
		Обозначим за $\mfrk D$ следующее отображение:
		$$
			\mfrk D = \sum\limits_{i=1}^n h_i \cfrac {\partial} {\partial x_i},
		$$
		где $h_i$ -- $i$-ая компонента фиксированного вектора $h = (h_1, \ldots, h_n)^T$.
		
		Если функция $f \in D^k$, то 
		\begin{multline*}
			\mfrk{D}^k f = \left( \sum\limits_{i=1}^n h_i \cfrac{\partial} {\partial x_i} \right)^k f =\\
			= \left( \sum\limits_{\substack{\alpha_i \ge 0,\ i=1,\ldots,n \\ \alpha_1+\dots\alpha_n=k}} \cfrac {k!}{\alpha_1!\dots \alpha_n!} \prod\limits_{i=1}^n h_i^{\alpha_i}\cfrac {\partial^{\alpha_i}}{\partial x_i^{\alpha_i}} \right)f =\\
			= \left(\sum\limits_{|\alpha|=k} \cfrac{|\alpha|!}{\alpha!} h^{\alpha} \cfrac {\partial^{\alpha}}{\partial x^{\alpha}}\right) f.
		\end{multline*}
	\end{example}

	\begin{theorem}[Формула Тейлора с остаточным членом в форме Пеано]
		Пусть $f\in D^k(a)$, где $a\in U \subset \mbb R^n$.
		
		Тогда
		$$
			f(a+h) = \sum\limits_{i=0}^k \left( \sum\limits_{|\alpha|=i} \cfrac {1}{\alpha!}  \cfrac {\partial^\alpha f(a)}{\partial x^\alpha} h^\alpha\right) + o(\|h\|^k),
		$$
		
		где $o(\|h\|^k) \to 0$ при $\|h\|\to 0$.
	\end{theorem}
	\begin{Proof}
		Доказательство будем проводить индукцией по $k$.
		
		База индукции. $k=1$ -- определение дифференцируемой функции.
		
		Шаг индукции. $k \Rightarrow k+1$. 
		
		Рассмотрим функцию $$\psi(h) = f(a+h) - \sum\limits_{i=0}^k \left( \sum\limits_{|\alpha|=i} \cfrac {1}{\alpha!}  \cfrac {\partial^\alpha f(a)}{\partial x^\alpha} h^\alpha\right).$$
		
		Ее производная по $h_j$ выглядит следующим образом:
		$$
			\cfrac {\partial \psi(h)} {\partial x_j} = \cfrac {\partial f} {\partial x_j} + \sum\limits_{i=0}^{k-1} \left( \sum\limits_{|\beta|=i} \cfrac {1}{\beta!}  \cfrac {\partial^\beta }{\partial x^\beta} \cfrac {\partial f(a)}{\partial x_j} h^\beta\right)
		$$
		
		При этом верно разложение $\alpha = \beta + \delta_{ij}$, где $\delta_{ij} = (0, \dots, \underset{j}{1}, \dots, 0)$ и $|\beta| = k-1$.
		
		По предположению индукции, $\cfrac {\partial \psi(h)} {\partial x_j} = o(\|h\|^{k-1})$.
		
		По лемме о приращении, имеем
		$$
			|\psi(h)-\psi(0)\| \le o(\|h\|^{k-1})\|h\|.
		$$
		
		Следовательно, $\psi(h) - \psi(0) = o(\|h\|^k)$, значит, 
		$$
			f(a+h) = \sum\limits_{i=0}^k \left( \sum\limits_{|\alpha|=i} \cfrac {1}{\alpha!}  \cfrac {\partial^\alpha f(a)}{\partial x^\alpha} h^\alpha\right) + o(\|h\|^k)
		$$
	\end{Proof}

	\begin{mdef}
		$
		\sum\limits_{i=0}^k \left( \sum\limits_{|\alpha|=i} \cfrac {1}{\alpha!}  \cfrac {\partial^\alpha f(a)}{\partial x^\alpha} h^\alpha\right)
		$ -- полином Тейлора порядка $k$ в точке $a\in U$ функции $f\in D^k(a)$.
	\end{mdef}

	\begin{lemma}
		$h^\alpha = o(\|h\|^k)$ при $\|h\|\to0$ тогда и только тогда, когда $k < |\alpha|$.
	\end{lemma}

	\begin{theorem}[Единственность полинома Тейлора]
		Пусть существует полином 
		$$
			P = \sum \limits_{i=0}^k \sum\limits_{|\alpha|=i} P_\alpha h^\alpha
		$$
		такой, что 
		$$
			f(a+h) - P = o(\|h\|^k)
		$$ при $\|h\|\to 0$.
		
		Тогда $P$ -- полином Тейлора.
	\end{theorem}
	
	\begin{corollary}
		Пусть $f\in D^k(a)$ и $$\cfrac {\partial^\alpha f(a)} {\partial x^\alpha} = 0$$ при $|\alpha| \le k$.
		
		Тогда $f(a+h) = o(\|h\|^k)$ при $\|h\|\to0$.
	\end{corollary}
	
	\section{Дифференциалы высших порядков}
	Пусть $f = f(x_1, \ldots, x_n)$ -- дифференцируемая функция.
	Тогда дифференциал функции $f$ имеет вид
	$$
		df = \cfrac {\partial f} {\partial x_1} dx_1 + \dots + \cfrac {\partial f} {\partial x_n}.	
	$$
	
	Обозначим за $\delta$ приращение.
	\begin{mdef}
		Значение $\delta(df)$ при $\delta x_i = dx_i$ для всех $i = 1,\ldots,n$ называется вторым дифференциалом функции $f$ и обозначается $d^2 f$.
		$$
			d^2 f = \delta \left. \left( \sum \limits_{i=1}^n \cfrac {\partial f}{\partial x_i}dx_i \right) \right|_{\substack{\delta x_i = dx_i, \\ i = 1,\ldots, n}}
		$$ 	
	\end{mdef}
	\begin{mdef}
		$k$-кратный дифференциал функции $f$ определяется следующим образом:
		$$
			\left.\delta(d^{k-1}f)\right|_{\substack{\delta x_i = dx_i, \\ i = 1,\ldots, n}} = d^k f.
		$$
	\end{mdef}

	\begin{theorem}
		Пусть функция $f=f(x_1, \ldots, x_n)$ дважды дифференцируема, тогда ее второй дифференциал имеет вид
		\begin{enumerate}
			\item Если $x_1, \ldots, x_n$ -- независимые переменные, то
			$$
				d^2f = \sum\limits_{i=1}^n \sum\limits_{j=1}^n \cfrac {\partial^2 f} {\partial x_j \partial x_i} dx_j dx_i.
			$$
			\item Если $x_i = x_i(\xi_1, \ldots, \xi_m)$ для всех $i = 1,\ldots, n$ дважды дифференцируемые функции, и $f(x_1, \ldots, x_n)$ -- дважды дифференцируемая функция, тогда
			$$
				d^2 f = \sum\limits_{i=1}^n \sum\limits_{j=1}^n \cfrac {\partial^2 f} {\partial x_j \partial x_i} dx_j dx_i +  \sum\limits_{k=1}^n \cfrac {\partial f} {\partial x_k} \sum\limits_{i=1}^m \sum\limits_{j=1}^m \cfrac {\partial^2 x_k} {\partial \xi_j \partial \xi_i} d\xi_j d\xi_i,
			$$ в более компактной форме:
			$$
				d^2 f = \left(\cfrac {\partial f} {\partial x_1} dx_1 + \dots +\cfrac {\partial f} {\partial x_n} dx_n\right)^2 + \sum \limits_{k=1}^n \cfrac {\partial f} {\partial x_k} d^2 x_k.
			$$
		\end{enumerate}
	\end{theorem}
	\begin{remark}
		Второй дифференциал функции $f$ от независимых переменных $x_1, \ldots, x_n$ $$d^2 f = \left(\cfrac {\partial f} {\partial x_1} dx_1 + \dots +\cfrac {\partial f} {\partial x_n} dx_n\right)^2$$ является симметричной квадратичной формой.
	\end{remark}

	\begin{statement} Пусть функция $f\colon U \to \mbb R$ $k$ раз дифференцируема в точке $a\in U$. Тогда 
		$$
		d^k f(a) = \sum_{\left| \alpha \right| = k} \cfrac{k!}{\alpha!} \cfrac{ \partial^{\left| \alpha \right|}f}{\partial x^{\alpha}}(a)(dx)^{\alpha}, \quad \text{где $(dx)^{\alpha} = (d{x_1})^{\alpha_1} \cdots (d{x_n})^{\alpha_n}$}
		$$
	\end{statement}
	
	\begin{corollary} Пусть функция $f\colon U \to \mbb R$ $k$ раз дифференцируема в точке $a\in U$. Тогда
		$$
		f(a+h) = f(a) + \cfrac 1 {1!}d f(a)\left< h\right> + \cfrac1 {2!} d^2 f(a) {\left< h \right>}^2 + \dots + \cfrac1{k!} d^k f(a) {\left< h \right>}^k + o(\| h \|^k)
		$$
	\end{corollary}
	
	\section{Экстремумы функций нескольких переменных}
	
	\section{Теорема о неявной функции}
	Для функции $F\colon U \to \mbb{R}^m$, где $U\subset \mbb{R}^{n+m} = \mbb{R}^n \times \mbb{R}^m$, то есть запись $(a, b) \in U$ означает, что $a\in \mbb R^n$, $b \in \mbb R^m$.
	Определим $F_a\colon U_a \to \mbb R^m$, где $U_a = \{y\in\mbb R^m\mid (a,y)\in U\}$, то есть $F_a(y) = F(a,y)$ и $F_b\colon U_b \to \mbb R^m$ аналогично ($F_b(x) = F(x, b)$).
	
	\begin{theorem}[О неявной функции]
		Пусть $F\colon U \to \mbb{R}^m$, где $U\subset \mbb{R}^{n+m}$, $U\in \mcl N[(a,b)]$ -- некоторая окрестность точки $(a,b)$. 
		
		Если выполнены следующие условия:
		\begin{enumerate}
			\item $F(a,b)=0$,
			\item \label{cond1} $F\in \mcl C^1 [(a,b)]$, то есть $F$ непрерывно дифференцируема в некоторой окрестности точки $(a,b)$,
			\item $dF_a(b)$ -- невырожденный оператор ($\det(dF_a(b))\not= 0$),
		\end{enumerate}
		то найдутся окрестности $V \in \mcl N(a)$, $V \subset \mbb R^n$, и $W \in \mcl N(b)$, $W \subset \mbb R^m$, и функция $f\colon V \to\mbb R^m$, $f\in\mcl D(V)$, что
		\begin{enumerate}
			\item $F(x,y) = 0 \Longleftrightarrow y=f(x)$ для $x\in V, y \in W$,
			\item $df(x) = - (dF_x(y))^{-1}dF_y(x)$, где $y = f(x)$.
		\end{enumerate}
	\end{theorem}	
	\begin{Proof}
		Обозначим $A = dF_a(b)$ и определим отображение $G_x(y) = y - A^{-1}F(x,y)$, $G_x\colon U_x \to \mbb R^m$, где $U_x = \{y\mid (x,y)\in U\}$. Проверим, что $G_x(y) = y \Leftrightarrow F(x,y) = 0$.
		\begin{equation}\label{nepodvizh}
		G_x(y) = y - A^{-1}F(x,y) =y \Leftrightarrow A^{-1}F(x,y) = 0 \Leftrightarrow F(x,y) = 0.
		\end{equation}
		
		Но подождите, а что за странная функция $G_x(y)$? Попробуем понять, как можно не заучивать это все дело. Смотрим <<за руками>>.
		
		Пусть функция $y=y(x)$ такая, что $F(x, y(x)) \equiv 0$. Тогда
		\begin{multline*}
		Ay(x) \equiv Ay(x) \Leftrightarrow Ay(x) - F(x, y(x)) \equiv Ay(x) \Leftrightarrow\\
		\Leftrightarrow A^{-1}(Ay(x) - F(x, y(x)) ) \equiv y(x)\ (\text{так как $A$ -- невырожден})\Leftrightarrow\\
		\Leftrightarrow y - A^{-1}F(x,y) \equiv y(x).
		\end{multline*}
		
		Теперь чуть понятнее откуда эта $G_x(y)$ берется, я надеюсь. Продолжаем.
		
		Мы хотим доказать, что найдутся $\delta_1 > 0$, $\delta_2 > 0$ такие, что при $\|x-a\|\le\delta_1$ отображение $G_x(y)$ является сжимающим на $\|y-b\|\le\delta_2$.
		
		Возьмем дифференциал от $G_x(y)$.
		$$
		dG_x(y) = d(y - A^{-1}F(x,y)) = E - A^{-1}dF(x,y).
		$$
		
		%Так как исходная функция $F$ была класса $\mcl C^1$, в точке $(a,b)$, то и $G_x(y) \in \mcl C^1(b)$. 
		
		Из непрерывности производных $F$ следует, что для любого $\varepsilon > 0$ найдутся $\delta_1, \delta_2 > 0$ такие, что $dF_x(y) = dF_a(b) + B_x(y)$, где $\|B_x(y)\|<\varepsilon$ при $\|x-a\|_1 \le \delta_1$ и $\|y-b\|_1 \le \delta_2$. Как это понимать? Если мы достаточно близки к точке $(a, b)$, то, по непрерывности, $dF_x(y)$ <<приближается>> к $dF_a(b)$.
		Тогда 
		\begin{multline*}
		\|dG_x(y)\|= \|E - A^{-1}dF(x,y)\| = \|A^{-1}(A - dF(x,y))\| \le\\
		\le \|A^{-1}\|\|A - dF(x,y)\| = \|A^{-1}\|\|B_x(y)\|.
		\end{multline*}
		
		Можно выбрать $\varepsilon > 0$ (от которого строится оценка на норму $B_x(y)$) так, что все частные производные $G_x(y)$ не превышали $\frac 1 2$ при $\|x-a\|_1 \le \delta_1$ и $\|y-b\|_1 \le \delta_2$.
		
		По лемме о приращении имеем, что $\|G_x(y)-G_x(z)\|_1\le \frac 1 2 \|y-z\|_1$ для всех $\|y-b\|_1 \le \delta_2$ и $\|z-b\|_1 \le \delta_2$
		
		Кроме того, мы имеем 
		\begin{multline*}
		\|G_x(y) - b\|_1 =  \|G_x(y)  - G_x(b) + G_x(b) - G_a(b)\|_1 \le \\
		\le \|G_x(y)  - G_x(b)\|_1 + \|G_x(b) - G_a(b)\|_1 \le \\ 
		\le \frac 1 2 \|y - b\|_1 + \sup\|G_x(b)-G_a(b)\|_1 = \frac 1 2 \|y - b\|_1 + r
		\end{multline*}
		
		При этом 
		\begin{multline*} r = \sup\|G_x(b)-G_a(b)\|_1 = \sup\| y - A^{-1}F(x,b) - y + A^{-1}F(a,b)\|_1 =\\
		= \sup \|A^{-1}(F(x,b) - F(a,b))\|_1 \le \|A^{-1}\|\|B_x(y)\|
		\end{multline*}
		при $\|x-a\|_1 \le \delta_1$.
		
		А теперь мы найдем столь малый $\delta_2 > 0$, что $r < \frac {\delta_2} 2$. Обозначим эту окрестность точки $a$ за $V$. Тогда каждая функция $G_x$ при $x\in V$  переводит множество $W = \{y \mid \|y-b\|_1 \le \delta_2\}$ в себя.
		
		Очень важно то, что знак неравенства в окрестностях -- нестрогий, иначе бы наша окрестность точки $y$ была неполной, так как она не компактна, и мы бы не смогли применить теорему о неподвижной точке.
		
		Мы проверили, что $G_x$ явлется сжимающим отображением на $W$, тогда, по теореме о неподвижной точке, для каждого $x\in V$ найдется единственная точка $z_x\in W$ такая, что $G_x(z) = z_x$. Если кому-то непонятно, почему неподвижная точка -- то, что нам требуется, то посмотрите (\ref{nepodvizh}).
		
		Определим $f(x) = z_x$. Этим мы доказали существование неявной функции.
		
		\textbf{Докажем непрерывность неявной функции.}
		
		Рассмотрим произвольную точку $(x_0, y_0)\in V\times W$ ($x_0\in V, y_0\in W$). 
		Так как $dF_a(b)$ -- невырожденный оператор из $\mbb R^m$ в $\mbb R^m$, то его определитель не равен нулю, а определитель 
		$\left(\det A = \sum\limits_{\pi \in \mbb S_m}\text{sgn}(\pi)\prod\limits_{i=1}^{m}a_{i, \pi(i)}\right)$ -- непрерывная функция от элементов матрицы, то в некоторой окрестности точки $(a,b)$ оператор $dF_{x_0}(y_0)$ так же является невырожденным, так как, по условию \ref{cond1}) теоремы, $F\in \mcl C^1[(x_0, y_0)]$.
		
		Можем считать, что $V\times W$ входит в эту окрестность (если не входит, то уменьшим $\delta_1, \delta_2$).
		
		Итого, можем сделать так:
		\begin{multline*}
		F(x,y) - F(x_0, y_0) = dF(x_0, y_0)\left< x-x_0, y - y_0\right> + o(\|x-x_0\|_1 + \|y-y_0\|_1) =\\
		=dF_{x_0}(y_0)\left< y-y_0\right> + dF_{y_0}(x_0)\left< x-x_0\right> + \alpha(x,y)(\|x-x_0\|_1 + \|y-y_0\|_1),
		\end{multline*}
		где $\alpha(x,y)\to0$ при $x\to x_0$ и $y\to y_0$.
		
		Пусть $y=f(x)$ и $y_0 = f(x_0)$.
		Тогда
		\begin{multline*}
		F(x, f(x)) - F(x_0, f(x_0)) = 0 = dF_{x_0}(y_0)\left< f(x)-f(x_0)\right> +\\
		+ dF_{y_0}(x_0)\left< x-x_0\right> + \alpha(x,y)(\|x-x_0\|_1 + \|f(x)-f(x_0)\|_1).
		\end{multline*}
		
		Применим обратную матрицу $(dF_{x_0}(y_0))^{-1}$ к равенству.
		
		\begin{multline*}
		0 = \left< y-y_0\right> + (dF_{x_0}(y_0))^{-1}dF_{y_0}(x_0)\left< x-x_0\right> + \\
		+(dF_{x_0}(y_0))^{-1}\alpha(x,y)(\|x-x_0\|_1 + \|f(x)-f(x_0)\|_1)\Rightarrow\\
		\Rightarrow f(x)-f(x_0) = -(dF_{x_0}(y_0))^{-1}dF_{y_0}(x_0)\left< x-x_0\right> -\\
		- (dF_{x_0}(y_0))^{-1}\alpha(x,y)(\|x-x_0\|_1 + \|f(x)-f(x_0)\|_1).
		\end{multline*}
		
		Оценим $f(x)-f(x_0)$ по норме (мы же хотим доказать непрерывность $f$).
		\begin{multline*}
		\| f(x)-f(x_0) \|_1 = \|-(dF_{x_0}(y_0))^{-1}dF_{y_0}(x_0)\left< x-x_0\right> - \\
		-(dF_{x_0}(y_0))^{-1}\alpha(x,y)(\|x-x_0\|_1 + \|f(x)-f(x_0)\|_1) \|_1 \le \\
		\le \|(dF_{x_0}(y_0))^{-1}dF_{y_0}(x_0)\left< x-x_0\right> \|_1 + \\
		+\|(dF_{x_0}(y_0))^{-1}\alpha(x,y)(\|x-x_0\|_1 + \|f(x)-f(x_0)\|_1) \|_1 \le \\
		\le \|(dF_{x_0}(y_0))^{-1}\|_1\|dF_{y_0}(x_0)\left< x-x_0\right> \|_1 + \\
		+\|(dF_{x_0}(y_0))^{-1}\|_1\|\alpha(x,y)\|_1(\|x-x_0\|_1 + \|f(x)-f(x_0)\|_1) 
		\end{multline*}
		
		Так как $\alpha(x,y)\to0$ при $x\to x_0$ и $y\to y_0$, то в найдется малая окрестность точки $(x_0, y_0)$ такая, что $$\|(dF_{x_0}(y_0))^{-1}\|_1\|\alpha(x,y)\|_1 < \frac 1 2$$ и, значит, 
		$ \frac 1 2 \|f(x)-f(x_0)\|_1 < C \|x-x_0\|_1$, то есть функция непрерывна (мы доказали, что функция непрерывна в $x_0$, но $x_0$ -- произвольная точка $V$).
		
		Теперь \textbf{остается доказать дифференцируемость}. Последний пункт теоремы, а потом ч.т.д., как говорится.
		
		Смотрим внимательно сюда.
		\begin{multline*}
		f(x)-f(x_0) = -(dF_{x_0}(y_0))^{-1}dF_{y_0}(x_0)\left< x-x_0\right> - \\
		-(dF_{x_0}(y_0))^{-1}\alpha(x,y)(\|x-x_0\|_1 + \|f(x)-f(x_0)\|_1)
		\end{multline*}
		Положим $\beta(x,y) = -(dF_{x_0}(y_0))^{-1}\alpha(x,y) \|f(x)-f(x_0)\|_1$, при этом $$\|\beta(x,y)\|_1 \le C \|(dF_{x_0}(y_0))^{-1} \|_1\|\alpha(x,y)\|_1\|x-x_0\| \le K \|x-x_0\|_1$$.
		
		То есть
		\begin{multline*}
		f(x)-f(x_0) = -(dF_{x_0}(y_0))^{-1}dF_{y_0}(x_0)\left< x-x_0\right> -\\
		- (dF_{x_0}(y_0))^{-1}\alpha(x,y)\|x-x_0\|_1 + \beta(x,y).
		\end{multline*}
		
		Функция $f$ дифференцируема и $df(x_0) = -(dF_{x_0}(y_0))^{-1}dF_{y_0}(x_0)$.
	\end{Proof}

	\begin{corollary}[О гладкости неявной функции]
		Если при выполнении условий теоремы о неявной функции потребовать, что бы $F\in\mcl C^r[(a,b)]$, то и неявная функция $f$ будет принадлежать классу $\mcl C^r$ в области определения $V$.
	\end{corollary}
	\begin{Proof}
		Пусть $y=f(x)$ -- неявная функция такая, что $F(x, y) = 0$. 
		
		База индукции. $r=1$ -- утверждение теоремы о неявной функции.
		
		Шаг индукции. $r\Rightarrow r+1$.
		
		Предполагается, что $F \in \mcl C^{r+1} [(a,b)]$ и $f \in \mcl C^r[(a)]$.
		
		Покажем рассуждения для второй производной -- последующие аналогично, но вычислений больше. 
		
		Исходя из формулы дифференциала неявной функции, мы имеем:
		$$
			\cfrac {\partial F} {\partial x} + \cfrac {\partial F} {\partial y} \cfrac {\partial y} {\partial x} = 0.
		$$
		
		Отсюда, 
		$$
			\cfrac {\partial y} {\partial x} = - \left(\cfrac {\partial F} {\partial y}\right)^{-1}\cfrac {\partial F} {\partial x}
		$$
		
		Продифференцируем по $x$ еще раз.
		$$
			\cfrac {\partial^2 y} {\partial x^2} = - \left(\cfrac {\partial F} {\partial y}\right)^{-1}\left( \cfrac {\partial^2 F} {\partial x^2} + \cfrac {\partial^2 F} {\partial y \partial x}\right) + \left(\cfrac {\partial F} {\partial y}\right)^{-2} \left( \cfrac {\partial^2 F} {\partial y \partial x} + \cfrac {\partial^2 F} {\partial y^2} \cfrac {\partial y} {\partial x}\right)
		$$
		
		Из этого равенства можно выразить $\cfrac {\partial^2 y} {\partial x^2}$, которая будет зависеть от второй производной $F$ и $\cfrac {\partial y} {\partial x}$, которые непрерывны по предположению. 
		
		Мы можем продолжать так выражать $k$-ые производные, где $k \le r+1$, которые будут зависеть он непрерывных функций, следовательно, сами будут непрерывны.
	\end{Proof}

	\begin{corollary}[Теорема об обратной функции]
		Пусть $f\colon U \to \mbb R^n$, $U\subset \mbb R^n$, $U \in \mcl N (b)$, $b\in \mbb R^n$. 
		
		Если верно, что 
		\begin{enumerate}
			\item $f\in \mcl C^1(b)$,
			\item $df(b)$ -- невырожденный оператор,
		\end{enumerate}
		то найдется $V\in \mcl N(f(b))$, $V \subset \mbb R^n$ в которой определена обратная функция $f^{-1}\colon V \to U$ такая, что $f^{-1}\in \mcl D(V)$ и $df^{-1}(x) = (df(y))^{-1}$, где $x=f(y)$.
	\end{corollary}
	\begin{Proof}
		Определим $F\colon \mbb R^n \times U \to \mbb R^n$ следующим образом: $F(x, y) = f(y) - x$. 
		Данная функция удовлетворяет условия теоремы о неявной функции в окрестности точки $(a,b)$, где $b=f(a)$. 
		
		Значит, найдется окрестность $V$ точки $f(b)$, и $W$ точки $a$, и  функция $g\colon V \to W$ такая, что $F(x, g(x)) = 0$ для любого $x\in V$. То есть $f(g(x)) = x$, при этом $dg(x) = (df(y))^{-1}$.
	\end{Proof}
	
	\section{Диффеоморфизмы}
	
	\begin{theorem}[Теорема о ранге]
		Пусть $U \subset \mbb R^n$ -- некоторая окрестность точки $x_0$.
		
		Отображение $f\colon U \to \mbb R^m$ является $\mcl C^r$-гладким, где $r \ge 1$, и $\text{rank}\,f(x) = k$ для всех $x \in U$.
		
		Тогда найдутся окрестности $V \subset \mbb R^n$ точки $x_0$ и $W \subset \mbb R^m$ точки $y_0 = f(x_0)$ и $\mcl C^r$-диффеоморфизмы $\varphi\colon V \to \mbb R^n$ и $\psi\colon W \to \mbb R^m$ такие, что отображение $h = \psi \circ f \circ \varphi^{-1}$ в окрестности $\varphi(V)$ имеет вид
		\begin{align}\label{theoremaORange}
			\mbb R^n \ni u = (u_1, \dots, u_k, \dots, u_n) \overset{h}{\mapsto} (u_1, \dots, u_k, \underbrace{0, \dots, 0}_{m-k}) = v \in \mbb R^m.
		\end{align}
	\end{theorem}
	\begin{Proof}
		Мы хотим построить отображение $h$, которое действовало бы в точности, как указано в \ref{theoremaORange}.
		
		Отразим зависимости функций на следующей коммутативной диаграмме:
		$$
		\begin{CD}
			\mbb R^n \supset U @>>{f}> W \subset \mbb R^m\\
			@AA{\varphi^{-1}}A @VV{\psi}V\\
			\mbb R^n \supset \varphi(V) @>>{h}> \mbb \psi(W) \subset \mbb R^m
		\end{CD}
		$$
		
		Введем следующее обозначение. 
		\begin{align*}
			&x^1 = (x_1, \dots, x_k)   	\quad  &y^1 = (y_1, \dots, y_k)\\
			&x^2 = (x_{k+1}, \dots x_n) \quad  &y^2 = (y_{k+1}, \dots, y_m)
		\end{align*}
		Аналогично будут определяться функции, но там уже размеры <<второй составляющей>> будут зависеть от контекста.
		
		И тогда фунция $f(x_1, \dots, x_n)$ будет представима в следующим виде:
		$$
			f(x_1, \dots, x_n) = \begin{pmatrix}
				f^1(x^1, x^2)\\
				f^2(x^2, x^2)
			\end{pmatrix}.
		$$
		
		Можем считать, что невырожденный минор размера $k\times k$ расположен в левом верхнем углу матрицы Якоби отображения $f$. Рассмотрим матрицу Якоби\footnote{Обозначение $\partial_{x^1} f^1$ подразумевает матрицу размера $k\times k$ вида $\left(\cfrac {\partial f_i} {\partial x_j}\right)_{\substack{i = 1, \dots, k\\ j = 1\dots k}}$. Аналогично определяются $\partial_{x^2} f^1, \partial_{x^1} f^2, \partial_{x^2} f^2$}:
		$$
			\mcl J_f = \begin{pmatrix}
				&\partial_{x^1} f^1 &\partial_{x^2} f^1\\
				&\partial_{x^1} f^2 &\partial_{x^2} f^2
			\end{pmatrix}
		$$
		
		Матрица $\partial_{x^1} f^1$ невырожденна. Рассмотрим отображение $\varphi$, задаваемое, следующим образом:
		\begin{align*}
			&\varphi^1(x^1, x^2) = f^1(x^1, x^2)\\
			&\varphi^2(x^1, x^2) = x^2.
		\end{align*}
		
		Рассмотрим матрицу Якоби этого отображения.
		$$
			\mcl J_{\varphi} = \begin{pmatrix}
				&\partial_{x^1} f^1 &\partial_{x^2} f^1\\
				&0 & E
			\end{pmatrix}
		$$
		Данная матрица невырождена и ее ранг равен $n$. По теореме о диффеоморфизмах, функция $\varphi$ -- $\mcl C^r$-гладкий диффеоморфизм в некоторой окрестности $V$ точки $x_0$ (мы пересекаем $U$ и $V$, чтобы получить итоговую окрестность, которая указана в формулировке) и определено обратное отображение 
		$$\varphi^{-1}\colon \varphi(V) \to V.$$
		
		Рассмотрим отображение $g\colon \varphi(V) \to f(U)$, определяемую следующим образом: $g= f \circ \varphi^{-1}$. Ранг данного отображения не больше, чем $k$.
		При этом
		\begin{align*}
		&g^1(u^1, u^2) = u^1\\
		&g^2(u^1, u^2) = g(u^1, u^2),
		\end{align*} при этом мы обозначаем вторую <<составляющую>> функции $g$ за $g$, так как первая часть -- тождественное отображение (надеюсь, это не вызовет путаницы).
		
		Рассмотрим матрицу Якоби отображения $g$.
		$$
			\mcl J_g = \begin{pmatrix}
				& E & 0\\
				& \partial_{u^1} g & \partial_{u^2} g
			\end{pmatrix}
		$$
		Из этого получаем, что $\partial_{u^2} g = 0$ (иначе могли бы получить матрицу ранга выше, чем $k$ -- противоречие). И можем считать, что $g = g(u^1)$, так как $\partial_{u^2} g = 0$.
		
		Определим отображение $\psi$ следующим образом:
		\begin{align*}
			&\psi^1(y^1, y^2) = y^1\\
			&\psi^2(y^1, y^2) = y^2 - g(y^1).
		\end{align*}
		
		Опять же рассмотрим матрицу Якоби отображения $\psi$:
		$$
			\mcl J_{\psi} = \begin{pmatrix}
				& E & 0\\
				&-\partial_{y^1} g & E
			\end{pmatrix}
		$$
		Якобиан $\psi$ равен $1$, следовательно, $\psi$ -- $\mcl C^r$-гладкий диффеоморфизм в некоторой окрестности $W$ (которую мы опять же пересечем с $f(U)$).
		
		Теперь мы можем составить требуемое отображение $h = \psi \circ f \circ \varphi^{-1}$. При этом
		\begin{align*}
		&h^1(u^1, u^2) = \psi^1 \circ f^1 \varphi^{-1} = \psi^1 \circ y^1 = y^1\\
		&h^2(u^1, u^2) = \psi^2 \circ f^2 \circ \varphi^{-1} = \psi^2 \circ g(u^1) = g(u^1) - g(u^1) = 0.
		\end{align*}
	\end{Proof}
	
	\section{Условные экстремумы} 