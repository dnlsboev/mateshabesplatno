%настройка документа
\documentclass[oneside, final]{book}
\usepackage[14pt]{extsizes}
\usepackage[utf8]{inputenc}
\usepackage[russian]{babel}
\frenchspacing

\binoppenalty=10000 
\relpenalty=10000 

%настройка страницы
\usepackage{vmargin}
\setpapersize{A4}
\setmarginsrb{2cm}{1.5cm}{1cm}{1.5cm}{1cm}{1cm}{0pt}{13mm}

%настройка переносов и красной строки
\usepackage{indentfirst}
\sloppy
%\fussy

%пакеты AMS
\usepackage{amsmath}
\usepackage{amsfonts}
\usepackage{amssymb}
\usepackage{amsthm}
\usepackage{amscd}

%дополнительные команды
\newcommand{\mbb}[1]{\mathbb{#1}}
\newcommand{\mcl}[1]{\mathcal{#1}}
\newcommand{\mfrk}[1]{\mathfrak{#1}}

%теоремы и тд с двойной нумерацией
\theoremstyle{plain}
\newtheorem{theorem}{Теорема}[chapter]
\newtheorem{lemma}[theorem]{Лемма}
\newtheorem{corollary}[theorem]{Следствие}
\newtheorem{statement}[theorem]{Утверждение}
\newtheorem{remark}[theorem]{Замечание}
\newtheorem{example}{Пример}[chapter]

\theoremstyle{definition}
\newtheorem{mdef}{Определение}[chapter]

%окружение доказательства
\newenvironment{Proof}[1][Доказательство] % имя окружения 
{\par\noindent{\bf #1. }} % команды для \begin 
{\hfill $\scriptstyle\qed$} % команды для \end 

\title{\textbf{Функции нескольких переменных}\\ Основные утверждения}
\author{Математический анализ, 3 семестр}
\date{ММФ НГУ\\ \vfill2019 год}

\begin{document}
	\maketitle
	
	\thispagestyle{empty}
	\vfill 
		Данный документ содержит основные теоремы и утверждения по функциям нескольких переменных и многообразиям.
		
		Выражаю свою благодарность Джозефу Лунгу за помощь при написании некоторых разделов.
		
	\vfill
	\hfill Сбоев Данил
	\vfill
	\newpage
	\thispagestyle{empty}
	\tableofcontents
	
	\input vs

	\input ffv
		
	\input manifolds
	
	\begin{thebibliography}{0}
		\bibitem{Greshnov} А.\,В.~Грешнов, Лекции по математическому анализу, 3 семестр, ММФ НГУ.
		\bibitem{Potapov} В.\,Н.~Потапов, Электронные лекции по математическому анализу, ММФ НГУ.
		\bibitem{Vodopyanov} С.\,К.~Водопьянов, Дифференциальное исчисление функций многих переменных, ММФ НГУ.
		\bibitem{HSE} А.\,М.~Красносельский, Конспект лектора, математический анализ, 1 курс, 4 модуль, 2017 год. 
		\bibitem{Shvedov} И.\,А.~Шведов, Компактный курс математического анализа, часть 2.
		\bibitem{Reshet} Ю.\,Г.~Решетняк, Курс математического анализа.
		\bibitem{Fixtengolc} Г.\,М.~Фихтенгольц, Курс дифференциального и интегрального исчисления, том 1.
		\bibitem{Rudin} У.~Рудин, Основы математического анализа.
		\bibitem{Dedonne} Ж.~Дьедонне, Основы современного анализа.
		\bibitem{Spivak} М.~Спивак, Математический анализ на многобразиях.
		\bibitem{Postnikov} М.\,М.~Постников, Лекции по геометрии. Семестр III. Гладкие многообразия.
	\end{thebibliography}
\end{document}
