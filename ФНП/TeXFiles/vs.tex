\chapter{Нормированные векторные пространства}	
	\begin{mdef}
		$\mcl V$ -- в. п-во над $\mbb R$, если
		\begin{align}
			&\forall x, y \in \mcl V \ x+y = y+x, \tag{\text{ВП.1}}\\
			&\forall x,y, z \in \mcl V \ (x+y)+z=x+(y+z), \tag{\text{ВП.2}}\\
			&\exists \vec 0\ \forall x \in \mcl V\ \vec 0 + x = x + \vec 0 = x, \tag{\text{ВП.3}}\\
			&\forall x \in \mcl V\ \exists -x \in \mcl V: \ x+(-x) = \vec 0, \tag{\text{ВП.4}}\\
			&\forall \alpha, \beta \in \mbb R\ \forall x \in \mcl V \ (\alpha+\beta)x = \alpha x+\beta x, \tag{\text{ВП.5}}\\
			&\forall \alpha, \beta \in \mbb R\ \forall x \in \mcl V\ (\alpha \beta) x = \alpha ( \beta x), \tag{\text{ВП.6}}\\
			&\forall \alpha \in \mbb R \ \forall x, y\in \mcl V\ \alpha(x+y) = \alpha x + \alpha y, \tag{\text{ВП.7}}\\
			&\exists 1 \in \mbb R \ \forall x \in \mcl V \ 1\cdot x = x. \tag{\text{ВП.8}}
		\end{align}
	\end{mdef}
	\begin{mdef}
		$\|\cdot\| \colon \mcl V \to \mbb R_{\ge 0}$ -- норма, если
		\begin{align}
			&\forall x\in \mcl V\ \|x\| \ge 0, \tag{\text{Н.1}}\\
			&\forall x, y \in \mcl V\ \|x+y\| \le \|x\|+\|y\|, \tag{\text{Н.2}}\\
			&\forall \alpha \in \mbb R \ \forall x \in \mcl V\ \|\alpha x \| = |\alpha|\|x\|, \tag{\text{Н.3}}\\
			&\|x\|=0 \Leftrightarrow x = \vec 0. \tag{\text{Н.4}}
		\end{align}
	\end{mdef}
	\begin{example}
		Примеры норм:
		\begin{enumerate}
			\item В пространстве $\mbb R^n$ введем норму $\|x\|_p = \left( \sum \limits_{k=1}^n |x_k|^p \right)^{\frac 1 p}$ и $\|x\|_{\infty} = \underset{1 \le i \le n}{\max}\{|x_i|\}$;
			\item В пространстве $C[a,b]$ норма $\|f\|_{\infty} = \underset{x \in [a,b]}{\sup}\{|f(x)|\}$ и $\|f\|_{C[a,b]} = \int\limits_a^b |f|dx$.
			\item В пространстве ограниченных последовательностей $\{x_n\}$ определим норму $\|x\|_{\infty} = \underset{n \in \mbb N}{\sup}\{|x_n|\}$, где $x$ -- последовательность $\{x_n\}$.		
		\end{enumerate}
	\end{example}
	\begin{remark}
		На нормированном в. п-ве можно определить метрику следующим образом:
		$$
			\rho(x, y) = \|x-y\|.
		$$	
		
		И это действительно будет метрикой.
	\end{remark}

	\begin{mdef}
		Банахово пространство -- полное (относительно метрики из предыдущего замечания) нормированное векторное пространство.
	\end{mdef}

	\begin{mdef}
		$\left( \cdot, \cdot \right)\colon \mcl V \times \mcl V \to \mbb R$ -- скалярное произведение, если 
		\begin{align}
			&\forall x \in \mcl V\ (x, x) \ge 0, \tag{\text{СП.1}}\\
			&\forall x,y \in \mcl V \ (x,y) = (y,x), \tag{\text{СП.2}}\\
			&\forall \alpha, \beta \in \mbb R \ \forall x,y, z \in \mcl V\ (\alpha x + \beta y, z) = (\alpha x, z) + (\beta y, z), \tag{\text{СП.3}}\\
			&(x,x) = 0 \Leftrightarrow x = \vec 0. \tag{\text{СП.4}}
		\end{align}
	\end{mdef}

	\begin{mdef}
		Если $\mcl V$ -- в. пр-во и $(\cdot, \cdot)$ -- скалярное произведение, то $\left(\mcl V, (\cdot, \cdot)\right)$ -- евклидово пространство.
	\end{mdef}

	\begin{statement}
		В евклидовом пространстве можно ввести норму следующим образом:
		$$
			\|x\| = \sqrt{(x, x)}.
		$$
	\end{statement}

	\begin{mdef}
		Две нормы $\|\cdot\|_1$ и $\|\cdot\|_2$ билипшицево эквивалентны, если 
		$$
			\exists C > 0:\  \cfrac 1 C \|x\|_1 \le \|x\|_2 \le C \|x\|_1, \quad \forall x \in \mcl V 
		$$
	\end{mdef}

	\begin{theorem}
		Все нормы в $\mbb R ^n$ билипшицево эквивалентны.
	\end{theorem}
	\begin{Proof}
		Докажем, что любая норма $f$ билипшицево эквивалентна $\|\cdot\|_2$.
		
		\begin{multline*}
			f(x) = f\left(\sum\limits_{k=1}^n x_k e_k\right) \le \sum\limits_{k=1}^n |x_k| f(e_k) \underset{\text{Н-во К-Б-Ш}}{\le}\\ 
			\underset{\text{Н-во К-Б-Ш}}{\le}\left(\sum\limits_{k=1}^n |x_k|^2 \right)^{\frac 1 2} \left( \sum\limits_{k=1}^n f(e_k) ^ 2 \right)^{\frac 1 2} = C_1 \|x\|_2.
		\end{multline*}
		
		Обратим неравенство.
		
		$$
			f(x) = f\left(\frac {\|x\|_2} {\|x\|_2} f\right) = {\|x\|_2} f\left(\frac 1 {\|x\|_2} f\right) \ge {\|x\|_2} \underset{\|\xi\|_2 = 1}{f(\xi)} = {\|x\|_2} C_2.
		$$
	\end{Proof}
	\begin{corollary}
		Последовательность точек
		\begin{equation*}
			v^m = \begin{pmatrix}
			v^m_1 \\
			\vdots \\
			v^m_n
			\end{pmatrix}
		\end{equation*}
		сходится к точке 
		\begin{equation*}
			u = \begin{pmatrix}
				u_1 \\
				\vdots \\
				u_n
			\end{pmatrix}
		\end{equation*} 
		при $m\to\infty$ тогда и только тогда, когда $v_i^m \underset{m \to \infty}{\longrightarrow} u_i$ для всех $i=1,\ldots, n$.
	\end{corollary}

	\begin{mdef}
		Введем понятие отображения $F \colon \mbb R^n \to \mbb R^m$.
		$$
			F(x) = 	\begin{pmatrix}
				f_1(x) \\
				\vdots \\
				f_m(x)
			\end{pmatrix},
		$$ где $x \in U$ и $U$ -- область в $\mbb R^n$.
	\end{mdef}
	\begin{corollary}
		$$\lim \limits_{x\to a} F(x) = u \Leftrightarrow \lim\limits_{\eta \to a_i} f_i(\eta) = u_i \quad \text{для всех $i=1,\ldots, m$,} $$
	\end{corollary}
	где 
	$$
		u = \begin{pmatrix}
			u_1\\
			\vdots\\
			u_n
		\end{pmatrix}.
	$$